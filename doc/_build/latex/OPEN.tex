% Generated by Sphinx.
\def\sphinxdocclass{report}
\documentclass[letterpaper,10pt,english]{sphinxmanual}
\usepackage[utf8]{inputenc}
\DeclareUnicodeCharacter{00A0}{\nobreakspace}
\usepackage[T1]{fontenc}
\usepackage{babel}
\usepackage{times}
\usepackage[Bjarne]{fncychap}
\usepackage{longtable}
\usepackage{sphinx}
\usepackage{multirow}


\title{OPEN Documentation}
\date{March 08, 2014}
\release{0.0.3}
\author{João Faria}
\newcommand{\sphinxlogo}{}
\renewcommand{\releasename}{Release}
\makeindex

\makeatletter
\def\PYG@reset{\let\PYG@it=\relax \let\PYG@bf=\relax%
    \let\PYG@ul=\relax \let\PYG@tc=\relax%
    \let\PYG@bc=\relax \let\PYG@ff=\relax}
\def\PYG@tok#1{\csname PYG@tok@#1\endcsname}
\def\PYG@toks#1+{\ifx\relax#1\empty\else%
    \PYG@tok{#1}\expandafter\PYG@toks\fi}
\def\PYG@do#1{\PYG@bc{\PYG@tc{\PYG@ul{%
    \PYG@it{\PYG@bf{\PYG@ff{#1}}}}}}}
\def\PYG#1#2{\PYG@reset\PYG@toks#1+\relax+\PYG@do{#2}}

\expandafter\def\csname PYG@tok@gd\endcsname{\def\PYG@tc##1{\textcolor[rgb]{0.63,0.00,0.00}{##1}}}
\expandafter\def\csname PYG@tok@gu\endcsname{\let\PYG@bf=\textbf\def\PYG@tc##1{\textcolor[rgb]{0.50,0.00,0.50}{##1}}}
\expandafter\def\csname PYG@tok@gt\endcsname{\def\PYG@tc##1{\textcolor[rgb]{0.00,0.27,0.87}{##1}}}
\expandafter\def\csname PYG@tok@gs\endcsname{\let\PYG@bf=\textbf}
\expandafter\def\csname PYG@tok@gr\endcsname{\def\PYG@tc##1{\textcolor[rgb]{1.00,0.00,0.00}{##1}}}
\expandafter\def\csname PYG@tok@cm\endcsname{\let\PYG@it=\textit\def\PYG@tc##1{\textcolor[rgb]{0.25,0.50,0.56}{##1}}}
\expandafter\def\csname PYG@tok@vg\endcsname{\def\PYG@tc##1{\textcolor[rgb]{0.73,0.38,0.84}{##1}}}
\expandafter\def\csname PYG@tok@m\endcsname{\def\PYG@tc##1{\textcolor[rgb]{0.13,0.50,0.31}{##1}}}
\expandafter\def\csname PYG@tok@mh\endcsname{\def\PYG@tc##1{\textcolor[rgb]{0.13,0.50,0.31}{##1}}}
\expandafter\def\csname PYG@tok@cs\endcsname{\def\PYG@tc##1{\textcolor[rgb]{0.25,0.50,0.56}{##1}}\def\PYG@bc##1{\setlength{\fboxsep}{0pt}\colorbox[rgb]{1.00,0.94,0.94}{\strut ##1}}}
\expandafter\def\csname PYG@tok@ge\endcsname{\let\PYG@it=\textit}
\expandafter\def\csname PYG@tok@vc\endcsname{\def\PYG@tc##1{\textcolor[rgb]{0.73,0.38,0.84}{##1}}}
\expandafter\def\csname PYG@tok@il\endcsname{\def\PYG@tc##1{\textcolor[rgb]{0.13,0.50,0.31}{##1}}}
\expandafter\def\csname PYG@tok@go\endcsname{\def\PYG@tc##1{\textcolor[rgb]{0.20,0.20,0.20}{##1}}}
\expandafter\def\csname PYG@tok@cp\endcsname{\def\PYG@tc##1{\textcolor[rgb]{0.00,0.44,0.13}{##1}}}
\expandafter\def\csname PYG@tok@gi\endcsname{\def\PYG@tc##1{\textcolor[rgb]{0.00,0.63,0.00}{##1}}}
\expandafter\def\csname PYG@tok@gh\endcsname{\let\PYG@bf=\textbf\def\PYG@tc##1{\textcolor[rgb]{0.00,0.00,0.50}{##1}}}
\expandafter\def\csname PYG@tok@ni\endcsname{\let\PYG@bf=\textbf\def\PYG@tc##1{\textcolor[rgb]{0.84,0.33,0.22}{##1}}}
\expandafter\def\csname PYG@tok@nl\endcsname{\let\PYG@bf=\textbf\def\PYG@tc##1{\textcolor[rgb]{0.00,0.13,0.44}{##1}}}
\expandafter\def\csname PYG@tok@nn\endcsname{\let\PYG@bf=\textbf\def\PYG@tc##1{\textcolor[rgb]{0.05,0.52,0.71}{##1}}}
\expandafter\def\csname PYG@tok@no\endcsname{\def\PYG@tc##1{\textcolor[rgb]{0.38,0.68,0.84}{##1}}}
\expandafter\def\csname PYG@tok@na\endcsname{\def\PYG@tc##1{\textcolor[rgb]{0.25,0.44,0.63}{##1}}}
\expandafter\def\csname PYG@tok@nb\endcsname{\def\PYG@tc##1{\textcolor[rgb]{0.00,0.44,0.13}{##1}}}
\expandafter\def\csname PYG@tok@nc\endcsname{\let\PYG@bf=\textbf\def\PYG@tc##1{\textcolor[rgb]{0.05,0.52,0.71}{##1}}}
\expandafter\def\csname PYG@tok@nd\endcsname{\let\PYG@bf=\textbf\def\PYG@tc##1{\textcolor[rgb]{0.33,0.33,0.33}{##1}}}
\expandafter\def\csname PYG@tok@ne\endcsname{\def\PYG@tc##1{\textcolor[rgb]{0.00,0.44,0.13}{##1}}}
\expandafter\def\csname PYG@tok@nf\endcsname{\def\PYG@tc##1{\textcolor[rgb]{0.02,0.16,0.49}{##1}}}
\expandafter\def\csname PYG@tok@si\endcsname{\let\PYG@it=\textit\def\PYG@tc##1{\textcolor[rgb]{0.44,0.63,0.82}{##1}}}
\expandafter\def\csname PYG@tok@s2\endcsname{\def\PYG@tc##1{\textcolor[rgb]{0.25,0.44,0.63}{##1}}}
\expandafter\def\csname PYG@tok@vi\endcsname{\def\PYG@tc##1{\textcolor[rgb]{0.73,0.38,0.84}{##1}}}
\expandafter\def\csname PYG@tok@nt\endcsname{\let\PYG@bf=\textbf\def\PYG@tc##1{\textcolor[rgb]{0.02,0.16,0.45}{##1}}}
\expandafter\def\csname PYG@tok@nv\endcsname{\def\PYG@tc##1{\textcolor[rgb]{0.73,0.38,0.84}{##1}}}
\expandafter\def\csname PYG@tok@s1\endcsname{\def\PYG@tc##1{\textcolor[rgb]{0.25,0.44,0.63}{##1}}}
\expandafter\def\csname PYG@tok@gp\endcsname{\let\PYG@bf=\textbf\def\PYG@tc##1{\textcolor[rgb]{0.78,0.36,0.04}{##1}}}
\expandafter\def\csname PYG@tok@sh\endcsname{\def\PYG@tc##1{\textcolor[rgb]{0.25,0.44,0.63}{##1}}}
\expandafter\def\csname PYG@tok@ow\endcsname{\let\PYG@bf=\textbf\def\PYG@tc##1{\textcolor[rgb]{0.00,0.44,0.13}{##1}}}
\expandafter\def\csname PYG@tok@sx\endcsname{\def\PYG@tc##1{\textcolor[rgb]{0.78,0.36,0.04}{##1}}}
\expandafter\def\csname PYG@tok@bp\endcsname{\def\PYG@tc##1{\textcolor[rgb]{0.00,0.44,0.13}{##1}}}
\expandafter\def\csname PYG@tok@c1\endcsname{\let\PYG@it=\textit\def\PYG@tc##1{\textcolor[rgb]{0.25,0.50,0.56}{##1}}}
\expandafter\def\csname PYG@tok@kc\endcsname{\let\PYG@bf=\textbf\def\PYG@tc##1{\textcolor[rgb]{0.00,0.44,0.13}{##1}}}
\expandafter\def\csname PYG@tok@c\endcsname{\let\PYG@it=\textit\def\PYG@tc##1{\textcolor[rgb]{0.25,0.50,0.56}{##1}}}
\expandafter\def\csname PYG@tok@mf\endcsname{\def\PYG@tc##1{\textcolor[rgb]{0.13,0.50,0.31}{##1}}}
\expandafter\def\csname PYG@tok@err\endcsname{\def\PYG@bc##1{\setlength{\fboxsep}{0pt}\fcolorbox[rgb]{1.00,0.00,0.00}{1,1,1}{\strut ##1}}}
\expandafter\def\csname PYG@tok@kd\endcsname{\let\PYG@bf=\textbf\def\PYG@tc##1{\textcolor[rgb]{0.00,0.44,0.13}{##1}}}
\expandafter\def\csname PYG@tok@ss\endcsname{\def\PYG@tc##1{\textcolor[rgb]{0.32,0.47,0.09}{##1}}}
\expandafter\def\csname PYG@tok@sr\endcsname{\def\PYG@tc##1{\textcolor[rgb]{0.14,0.33,0.53}{##1}}}
\expandafter\def\csname PYG@tok@mo\endcsname{\def\PYG@tc##1{\textcolor[rgb]{0.13,0.50,0.31}{##1}}}
\expandafter\def\csname PYG@tok@mi\endcsname{\def\PYG@tc##1{\textcolor[rgb]{0.13,0.50,0.31}{##1}}}
\expandafter\def\csname PYG@tok@kn\endcsname{\let\PYG@bf=\textbf\def\PYG@tc##1{\textcolor[rgb]{0.00,0.44,0.13}{##1}}}
\expandafter\def\csname PYG@tok@o\endcsname{\def\PYG@tc##1{\textcolor[rgb]{0.40,0.40,0.40}{##1}}}
\expandafter\def\csname PYG@tok@kr\endcsname{\let\PYG@bf=\textbf\def\PYG@tc##1{\textcolor[rgb]{0.00,0.44,0.13}{##1}}}
\expandafter\def\csname PYG@tok@s\endcsname{\def\PYG@tc##1{\textcolor[rgb]{0.25,0.44,0.63}{##1}}}
\expandafter\def\csname PYG@tok@kp\endcsname{\def\PYG@tc##1{\textcolor[rgb]{0.00,0.44,0.13}{##1}}}
\expandafter\def\csname PYG@tok@w\endcsname{\def\PYG@tc##1{\textcolor[rgb]{0.73,0.73,0.73}{##1}}}
\expandafter\def\csname PYG@tok@kt\endcsname{\def\PYG@tc##1{\textcolor[rgb]{0.56,0.13,0.00}{##1}}}
\expandafter\def\csname PYG@tok@sc\endcsname{\def\PYG@tc##1{\textcolor[rgb]{0.25,0.44,0.63}{##1}}}
\expandafter\def\csname PYG@tok@sb\endcsname{\def\PYG@tc##1{\textcolor[rgb]{0.25,0.44,0.63}{##1}}}
\expandafter\def\csname PYG@tok@k\endcsname{\let\PYG@bf=\textbf\def\PYG@tc##1{\textcolor[rgb]{0.00,0.44,0.13}{##1}}}
\expandafter\def\csname PYG@tok@se\endcsname{\let\PYG@bf=\textbf\def\PYG@tc##1{\textcolor[rgb]{0.25,0.44,0.63}{##1}}}
\expandafter\def\csname PYG@tok@sd\endcsname{\let\PYG@it=\textit\def\PYG@tc##1{\textcolor[rgb]{0.25,0.44,0.63}{##1}}}

\def\PYGZbs{\char`\\}
\def\PYGZus{\char`\_}
\def\PYGZob{\char`\{}
\def\PYGZcb{\char`\}}
\def\PYGZca{\char`\^}
\def\PYGZam{\char`\&}
\def\PYGZlt{\char`\<}
\def\PYGZgt{\char`\>}
\def\PYGZsh{\char`\#}
\def\PYGZpc{\char`\%}
\def\PYGZdl{\char`\$}
\def\PYGZhy{\char`\-}
\def\PYGZsq{\char`\'}
\def\PYGZdq{\char`\"}
\def\PYGZti{\char`\~}
% for compatibility with earlier versions
\def\PYGZat{@}
\def\PYGZlb{[}
\def\PYGZrb{]}
\makeatother

\begin{document}

\maketitle
\tableofcontents
\phantomsection\label{index::doc}


Contents:


\chapter{Periodograms}
\label{api:periodograms}\label{api::doc}\label{api:welcome-to-open-s-documentation}\label{api:module-OPEN.periodograms}\index{OPEN.periodograms (module)}\phantomsection\label{api:module-OPEN.periodograms}\index{OPEN.periodograms (module)}

\section{Generalized Lomb-Scargle}
\label{api:generalized-lomb-scargle}\index{gls (class in OPEN.periodograms)}

\begin{fulllineitems}
\phantomsection\label{api:OPEN.periodograms.gls}\pysiglinewithargsret{\strong{class }\code{OPEN.periodograms.}\bfcode{gls}}{\emph{rv}, \emph{ofac=6}, \emph{hifac=1}, \emph{freq=None}, \emph{quantity='vrad'}, \emph{norm='HorneBaliunas'}, \emph{stats=False}, \emph{ext=True}}{}
Compute the Generalized Lomb-Scargle (GLS) periodogram.

This class implements the error-weighted Lomb-Scargle periodogram as
developed by {[}ZK09{]} using various possible normalizations.

The constructor takes a RVSeries instance (i.e. a rv curve) as 
first argument. As the algorithm is slow-ish, an implementation
in Fortran is available if the keyword `ext' is set to True in 
the constructor or globally. 
There is an optional \emph{freq} array, that can contain the 
frequencies on which to calculate the periodogram. If not provided
(....)
\begin{description}
\item[{Parameters}] \leavevmode\begin{description}
\item[{rv}] \leavevmode{[}RVSeries{]}
The radial velocity curve or any object providing the attributes
time, vrad and error which define the data.

\item[{ofac}] \leavevmode{[}int{]}
Oversampling factor (default=6).

\item[{hifac}] \leavevmode{[}float{]}
hifac * ``average'' Nyquist frequency is highest frequency for 
which the periodogram will be calculated (default=1).

\item[{freq}] \leavevmode{[}array, optional{]}
Contains the frequencies at which to calculate the periodogram.
If not given, a frequency array will be automatically generated.

\item[{quantity}] \leavevmode{[}string, optional{]}
For which quantity to calculate the periodogram. Possibilities are
`bis' or `fwhm' other than the default `vrad'.

\item[{norm}] \leavevmode{[}string, optional{]}
The normalization; either ``Scargle'', ``HorneBaliunas'', or 
``Cumming''. Default is ``HorneBaliunas''.

\item[{stats}] \leavevmode{[}boolean, optional{]}
Set True to obtain some statistical output (default is False).

\item[{ext}] \leavevmode{[}boolean, optional{]}
Use Fortran extension in the calculation (default is True)

\end{description}

\item[{Attributes}] \leavevmode\begin{description}
\item[{power}] \leavevmode{[}array{]}
The normalized power of the GLS.

\item[{freq}] \leavevmode{[}array{]}
The frequency array.

\item[{ofac}] \leavevmode{[}int{]}
The oversampling factor.

\item[{hifac}] \leavevmode{[}float{]}
The maximum frequency.

\item[{norm}] \leavevmode{[}string{]}
The normalization used.

\end{description}

\end{description}
\index{prob() (OPEN.periodograms.gls method)}

\begin{fulllineitems}
\phantomsection\label{api:OPEN.periodograms.gls.prob}\pysiglinewithargsret{\bfcode{prob}}{\emph{Pn}}{}
Probability of obtaining the given power.

Calculate the probability to obtain a power higher than
\emph{Pn} from the noise, which is assumed to be Gaussian.

\begin{notice}{note}{Note:}
This depends on the normalization
(see {[}ZK09{]} for further details).
\begin{itemize}
\item {} 
\emph{Scargle}:

\end{itemize}
\begin{gather}
\begin{split}exp(-Pn)\end{split}\notag\\\begin{split}\end{split}\notag
\end{gather}\begin{itemize}
\item {} 
\emph{HorneBaliunas}:

\end{itemize}
\begin{gather}
\begin{split}\left(1 - 2 \times \frac{Pn}{N-1} \right)^{(N-3)/2}\end{split}\notag\\\begin{split}\end{split}\notag
\end{gather}\begin{itemize}
\item {} 
\emph{Cumming}:

\end{itemize}
\begin{gather}
\begin{split}\left(1+2\times \frac{Pn}{N-3}\right)^{-(N-3)/2}\end{split}\notag\\\begin{split}\end{split}\notag
\end{gather}\end{notice}
\begin{description}
\item[{Parameters}] \leavevmode\begin{description}
\item[{Pn}] \leavevmode{[}float{]}
Power threshold.

\end{description}

\item[{Returns}] \leavevmode\begin{description}
\item[{Probability}] \leavevmode{[}float{]}
The probability to obtain a power equal or
higher than the threshold from the noise.

\end{description}

\end{description}

\end{fulllineitems}

\index{probInv() (OPEN.periodograms.gls method)}

\begin{fulllineitems}
\phantomsection\label{api:OPEN.periodograms.gls.probInv}\pysiglinewithargsret{\bfcode{probInv}}{\emph{Prob}}{}
Calculate minimum power for a given probability.

This function is the inverse of \emph{Prob(Pn)}.
Returns the minimum power for a given probability threshold Prob.
\begin{description}
\item[{Parameters}] \leavevmode\begin{description}
\item[{Prob}] \leavevmode{[}float{]}
Probability threshold.

\end{description}

\item[{Returns}] \leavevmode\begin{description}
\item[{Power threshold}] \leavevmode{[}float{]}
The minimum power for the given
false-alarm probability threshold.

\end{description}

\end{description}

\end{fulllineitems}


\end{fulllineitems}



\section{Bayesian Lomb-Scargle}
\label{api:bayesian-lomb-scargle}\index{bls (class in OPEN.periodograms)}

\begin{fulllineitems}
\phantomsection\label{api:OPEN.periodograms.bls}\pysiglinewithargsret{\strong{class }\code{OPEN.periodograms.}\bfcode{bls}}{\emph{rv}, \emph{ofac=6}, \emph{hifac=40}, \emph{freq=None}, \emph{quantity='vrad'}, \emph{stats=False}}{}
Compute the Bayesian Lomb-Scargle (BLS) periodogram.

This class implements the bayesian Lomb-Scargle periodogram as
developed by Bretthorst (2000, 2001). This corresponds to the 
bayesian expression for the posterior p(f\textbar{}D,I).

The constructor takes a RVSeries instance (i.e. a rv curve) as 
first argument. The algorithm is slow due to the required frequency
resolution; currently, only a Fortran implementation is available.
There is an optional \emph{freq} array, that can contain the 
frequencies on which to calculate the periodogram. If not provided
(....)
\begin{description}
\item[{Parameters}] \leavevmode\begin{description}
\item[{rv}] \leavevmode{[}RVSeries{]}
The radial velocity curve or any object providing the attributes
time, vrad and error which define the data.

\item[{ofac}] \leavevmode{[}int{]}
Oversampling factor (default=6).

\item[{hifac}] \leavevmode{[}float{]}
hifac * ``average'' Nyquist frequency is highest frequency for 
which the periodogram will be calculated (default=40).

\item[{freq}] \leavevmode{[}array, optional{]}
Contains the frequencies at which to calculate the periodogram.
If not given, a frequency array will be automatically generated.

\item[{quantity}] \leavevmode{[}string, optional{]}
For which quantity to calculate the periodogram. Possibilities are
`bis' or `fwhm' other than the default `vrad'.

\item[{stats}] \leavevmode{[}boolean, optional{]}
Set True to obtain some statistical output (default is False).

\end{description}

\item[{Attributes}] \leavevmode\begin{description}
\item[{power}] \leavevmode{[}array{]}
The normalized power of the GLS.

\item[{freq}] \leavevmode{[}array{]}
The frequency array.

\end{description}

\end{description}
\index{prob() (OPEN.periodograms.bls method)}

\begin{fulllineitems}
\phantomsection\label{api:OPEN.periodograms.bls.prob}\pysiglinewithargsret{\bfcode{prob}}{\emph{Pn}}{}
Probability of obtaining the given power.

Calculate the probability to obtain a power higher than
\emph{Pn} from the noise, which is assumed to be Gaussian.

\begin{notice}{note}{Note:}
Normalization
(see {[}ZK09{]} for further details).
\begin{itemize}
\item {} 
\emph{Scargle}:

\end{itemize}
\begin{gather}
\begin{split}exp(-Pn)\end{split}\notag\\\begin{split}\end{split}\notag
\end{gather}\begin{itemize}
\item {} 
\emph{HorneBaliunas}:

\end{itemize}
\begin{gather}
\begin{split}\left(1 - 2 \times \frac{Pn}{N-1} \right)^{(N-3)/2}\end{split}\notag\\\begin{split}\end{split}\notag
\end{gather}\begin{itemize}
\item {} 
\emph{Cumming}:

\end{itemize}
\begin{gather}
\begin{split}\left(1+2\times \frac{Pn}{N-3}\right)^{-(N-3)/2}\end{split}\notag\\\begin{split}\end{split}\notag
\end{gather}\end{notice}
\begin{description}
\item[{Parameters}] \leavevmode\begin{description}
\item[{Pn}] \leavevmode{[}float{]}
Power threshold.

\end{description}

\item[{Returns}] \leavevmode\begin{description}
\item[{Probability}] \leavevmode{[}float{]}
The probability to obtain a power equal or
higher than the threshold from the noise.

\end{description}

\end{description}

\end{fulllineitems}

\index{probInv() (OPEN.periodograms.bls method)}

\begin{fulllineitems}
\phantomsection\label{api:OPEN.periodograms.bls.probInv}\pysiglinewithargsret{\bfcode{probInv}}{\emph{Prob}}{}
Calculate minimum power for given probability.

This function is the inverse of \emph{Prob(Pn)}.
Returns the minimum power for a given probability threshold Prob.
\begin{description}
\item[{Parameters}] \leavevmode\begin{description}
\item[{Prob}] \leavevmode{[}float{]}
Probability threshold.

\end{description}

\item[{Returns}] \leavevmode\begin{description}
\item[{Power threshold}] \leavevmode{[}float{]}
The minimum power for the given
false-alarm probability threshold.

\end{description}

\end{description}

\end{fulllineitems}


\end{fulllineitems}



\chapter{Commands}
\label{commands:commands}\label{commands::doc}
At its core, OPEN is just an IPython shell enhanced with a few custom magic functions. You might be familiar with magic functions, things like \code{\%cd}, \code{\%run} or \code{\%timeit}. These can often be called without the leading `\%' from within IPython and offer a convenient way to execute common functions which have always the same type of parameters (or no parameters at all). Thus, by creating new magic functions we are basically adding custom commands to the shell. This is in addition to everything that is built-in to both Python and IPython so that all batteries are included; all the power of Python is there.

Currently, these are the commands available in \emph{OPEN}


\section{read}
\label{commands:read}
Read files with RV measurements.

\begin{Verbatim}[commandchars=\\\{\}]
Usage:
    read \PYGZlt{}file\PYGZgt{}...
    read \PYGZlt{}file\PYGZgt{}... [\PYGZhy{}d] [\PYGZhy{}\PYGZhy{}skip=\PYGZlt{}sn\PYGZgt{}] [\PYGZhy{}v]
    read \PYGZhy{}h \textbar{} \PYGZhy{}\PYGZhy{}help
Options:
    \PYGZhy{}d                  Set this as default system.
    \PYGZhy{}v \PYGZhy{}\PYGZhy{}verbose        Verbose output about data just read.
    \PYGZhy{}\PYGZhy{}skip=\PYGZlt{}sn\PYGZgt{}         How many header lines to skip [default: 0].
    \PYGZhy{}h \PYGZhy{}\PYGZhy{}help           Show this help message.
\end{Verbatim}


\section{plot}
\label{commands:plot}
Plot various quantities.

\begin{Verbatim}[commandchars=\\\{\}]
Usage:
    plot obs
    plot (fwhm \textbar{} rhk \textbar{} s \textbar{} bis \textbar{} contrast)
    plot \PYGZhy{}n SYSTEM
    plot \PYGZhy{}h \textbar{} \PYGZhy{}\PYGZhy{}help
Options:
    \PYGZhy{}n SYSTEM   Specify name of system (else use default)
    \PYGZhy{}h \PYGZhy{}\PYGZhy{}help   Show this help message
\end{Verbatim}


\section{per}
\label{commands:per}
Calculate periodograms.

\begin{Verbatim}[commandchars=\\\{\}]
Usage:
    per obs
    per (bis \textbar{} fwhm)
    per \PYGZhy{}n SYSTEM
    per (obs \textbar{} bis \textbar{} fwhm \textbar{} rhk \textbar{} resid) [\PYGZhy{}\PYGZhy{}gls\textbar{}\PYGZhy{}\PYGZhy{}bayes\textbar{}\PYGZhy{}\PYGZhy{}fast] [\PYGZhy{}v] [\PYGZhy{}\PYGZhy{}force] [\PYGZhy{}\PYGZhy{}hifac=\PYGZlt{}hf\PYGZgt{}] [\PYGZhy{}\PYGZhy{}ofac=\PYGZlt{}of\PYGZgt{}] [\PYGZhy{}\PYGZhy{}fap]
    per \PYGZhy{}h \textbar{} \PYGZhy{}\PYGZhy{}help
Options:
    \PYGZhy{}n SYSTEM     Specify name of system (else use default)
    \PYGZhy{}\PYGZhy{}gls         Calculate the Generalized Lomb\PYGZhy{}Scargle periodogram (default)
    \PYGZhy{}\PYGZhy{}bayes       Calculate the Bayesian periodogram
    \PYGZhy{}\PYGZhy{}fast        Calculate the Lomb\PYGZhy{}Scargle periodogram with fast algorithm
    \PYGZhy{}\PYGZhy{}force       Force recalculation
    \PYGZhy{}\PYGZhy{}hifac=\PYGZlt{}hf\PYGZgt{}  hifac * Nyquist is lowest frequency used [default: 40]
    \PYGZhy{}\PYGZhy{}ofac=\PYGZlt{}of\PYGZgt{}   Oversampling factor [default: 6]
    \PYGZhy{}\PYGZhy{}fap         Plot false alarm probabilities
    \PYGZhy{}v \PYGZhy{}\PYGZhy{}verbose  Verbose statistical output 
    \PYGZhy{}h \PYGZhy{}\PYGZhy{}help   Show this help message
\end{Verbatim}


\section{mod}
\label{commands:mod}
Define the model that will be adjusted to the data.

\begin{Verbatim}[commandchars=\\\{\}]
Usage:
    mod [k\textless{}n\textgreater{}] [d\textless{}n\textgreater{}]
Options:
    k\textless{}n\textgreater{}    Number of keplerian signals
    d\textless{}n\textgreater{}    Degree of polynomial drift
\end{Verbatim}


\section{restrict}
\label{commands:restrict}
Select data based on date, SNR or RV accuracy.

\begin{Verbatim}[commandchars=\\\{\}]
Usage:
    restrict [(err \PYGZlt{}maxerr\PYGZgt{})]
    restrict [(jd \PYGZlt{}minjd\PYGZgt{} \PYGZlt{}maxjd\PYGZgt{})]
    restrict [(year \PYGZlt{}yr\PYGZgt{})]
    restrict [(years \PYGZlt{}yr1\PYGZgt{} \PYGZlt{}yr2\PYGZgt{})]
    restrict \PYGZhy{}\PYGZhy{}gui
Options:
    \PYGZhy{}\PYGZhy{}gui         Restrict data using a graphical interface (experimental)
\end{Verbatim}


\section{killall}
\label{commands:killall}
Close all plot windows.


\chapter{Indices and tables}
\label{index:indices-and-tables}\begin{itemize}
\item {} 
\emph{genindex}

\item {} 
\emph{modindex}

\item {} 
\emph{search}

\end{itemize}


\renewcommand{\indexname}{Python Module Index}
\begin{theindex}
\def\bigletter#1{{\Large\sffamily#1}\nopagebreak\vspace{1mm}}
\bigletter{o}
\item {\texttt{OPEN.periodograms}}, \pageref{api:module-OPEN.periodograms}
\end{theindex}

\renewcommand{\indexname}{Index}
\printindex
\end{document}
